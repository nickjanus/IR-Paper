\documentclass{sig-alternate}
\usepackage{graphicx}
\graphicspath{{./figures/}}

\begin{document}
%
% --- Author Metadata here ---
%\conferenceinfo{WOODSTOCK}{'97 El Paso, Texas USA}
\CopyrightYear{2014} % Allows default copyright year (20XX) to be over-ridden - IF NEED BE.
%\crdata{0-12345-67-8/90/01}  % Allows default copyright data (0-89791-88-6/97/05) to be over-ridden - IF NEED BE.
% --- End of Author Metadata ---

\title{Improving Hashtag Comprehension with Search and Text Summarization
%  \titlenote{(Produces the permission block, and copyright information). For use with SIG-ALTERNATE.CLS. Supported by ACM.}
}

\numberofauthors{3}
\author{
% 1st. author
\alignauthor
John Lanchantin\\
       \affaddr{University of Virginia}\\
       \affaddr{85 Engineer's Way}\\
       \affaddr{Charlottesville, VA 22904-4740}\\
       \email{jjl5sw@virginia.edu}
% 2nd. author
\alignauthor
Nicholas Janus\\
       \affaddr{University of Virginia}\\
       \affaddr{85 Engineer's Way}\\
       \affaddr{Charlottesville, VA 22904-4740}\\
       \email{ncj2ey@virginia.edu}
% 3rd. author
\alignauthor 
Weilin Xu\\
       \affaddr{University of Virginia}\\
       \affaddr{85 Engineer's Way}\\
       \affaddr{Charlottesville, VA 22904-4740}\\
       \email{xuweilin@virginia.edu}
}

\maketitle
\begin{abstract}
Posts on micro-blogging sites are often very hard to understand due to their informality. Hashtags represent one solution to this problem by acting as subject markers for posts. However, hashtags are often difficult to understand without reading through multiple posts or conversations. We attempt to solve hashtag comprehension problem by automatically understanding what hashtags mean, and displaying relevant documents or text from within those documents.
\end{abstract}

% A category with the (minimum) three required fields
\category{Information systems}{Information Retrieval}{Specialized Information Retrieval}
\keywords{Micro-blogging, Hashtag retrieval, hashtag prediction, hashtag comprehension}

\section{Introduction}
The Internet today, especially social network services such as twitter and Facebook, is filled with 'hashtags'. Hashtags are single tokens that use the character '\#' in front of the words, and are often composed of natural language n-grams abbreviations, or acronyms. The problem is that there is no structure to hashtags beyond the format of the '\#' character and no spaces, thus making it terribly difficult to understand them. Users often create hashtags that are slang, concatenations of many words, acronyms, or simply made up words.  \\
There are certain hashtags that are very obvious to understand. For example, \#baseball, \#android, \#christmas each give a very clear cut definition of what they are about. On the other hand, there are many hashtags that are difficult to understand. For example. \#NCTL14, \#twitterblades, #icymi, \#Saban14, are very difficult to understand because they are not real words and there is no hint of what they mean directly from the hashtag.\\
An important task is to be able to automatically understand the underlying meaning behind a trending topic on social media. By understanding the meaning behind hashtags, we can further analyze what people are talking about. Often times, it does not become clear what someone is talking about until the meaning of his/her hashtag is understood, and it is frequently difficult to understand the hashtag due to their informality.\\
In order to understand the meaning behind hashtags, we propose two possible solutions. Our solution consists of selecting proper keywords query terms from tweets on Twitter using the inquired hashtag, and searching on websites such as wikipedia, urbandictionary, news websites etc.
\\

\section{Related Work}
The research surrounding hashtags covers a variety of topics including hashtag retrieval\cite{efron:retrieval}, hashtag prediction\cite{khabiri:predict}\cite{tagspace} for social media posts, or sentiment analysis for either the hashtag itself or the contents of the enclosing post.  Although these systems rely on an implicit understanding of the hash tag's meaning, they do not attempt to export such semantics.  Attempts to deliver the meaning of hashtags rely on crowdsourcing or manual annotation.\\
Currently, twitter manually expands certain acronym hashtags (e.g. \#oitnb translates to ?orange is the new black?), but it does not explain what they mean. In addition, there are a few websites (e.g. tagdef.com) which attempt to explain hashtags by crowdsourcing definitions, but there is often no clear cut definition. Both of the aforementioned methods do a poor job at explaining trending, or newly defined hashtags because they rely on the meaning of a hashtag over a long period of time, or have to wait until users explain the hashtag. In addition, the content quality is not as good as webpage links that give a more clearly defined explanation.
Our approach will have a much greater coverage of hashtag understanding and will not rely on manual annotation for an interpretation of the tag's meaning.  By giving access to our system via a browser plugin, our service will also be much more accessible and available to users of micro-blogging platforms.\\

\section{Approach and Implementation}

The main idea of our approach is as follows. Based on all tweets that use the inquired hashtag, we create a query for a current search engine (Bing.com), and display relevant web pages and a short summary based on those web pages to the user. \\
The system flow works in the following way: the user issues a hashtag to find its meaning using a browser plugin, the system extracts and filters all tweets using that hashtag, generates a query based on the tweets, searches that query using bing, returns the links of the most relevant web pages, and generates a summarization of the text from the top web page. The links and summary are displayed to the user via the plugin. This is shown in figure \ref{fig:SystemArchitechture}. Each module is explained is explained in further detail in the following subsections.

\begin{figure}[h!]
     \fbox{\includegraphics[width=.46\textwidth]{SystemArchitechture}}
   \caption{System Architechture} \label{fig:SystemArchitechture}
\end{figure}


\subsection{Tweet Extraction}
The Tweet Extraction module returns the text of the N most recent tweets that use the inquired hashtag using the Twitter Search API (we use N = 500 in our case). One limitation of this method is that currently, the Twitter Search API  only allows tweets from past 6-9 days. This makes it difficult to extract the meaning of hashtags that may have been used a few times in the past few days, but the majority of the times it was used was more than 6-9 days ago. \\
An important aspect of the tweet extraction is filtering the tweets that the search API retrieves. Firstly, we search the API using a '-filter:retweets' extension so that retweets are not considered because the top N most recent tweets very well may be all retweets. Secondly, we remove all URLS and emoticons from the tweets. We filter these out because emoticons cannot be used in a search engine search, and we make the assumption that the URLS that are included in tweets can be directly clicked on if they are relevant, so adding them to the query will not provide any additional relevant documents. In addition, we attempt to filter spam tweets by not including the text of tweets that have more than two trending hashtags. We understand that certain trending hashtags are sometimes correlated and thus are rightfully used together in a tweet, but we observed that most of the time spammers will use many trending hashtags in their tweets so that their tweets are seen by more viewers.
We consider the notion of filtering out tweets that do not have the same case-sensitive hashtag (e.g. \#IAD refers to the Washington/Dulles international airport, but \#iAd refers to Apple's advertising platform). \textbf{We observed that this does not give great improvement}.\\
Finally, we implement a "retrieve related hashtags" submodule which retrieves the top 5 most common co-occurring hashtags with the original inquired hashtag. This is useful due to the fact that often times the most common co-occurring hashtags are often related to the inquired hashtag, and may give more information about the inquired hashtag than it alone. This is a trivial task, but the user cannot parse all of the N most recent tweets and find the most common co-occurring hashtags, so it proves useful to be implemented in the plugin. 


\subsection{Query Generation}
The main idea of the query generation module is to generate a query to search in Bing based on the text from the tweet extraction module. Our method generates a query based on the high-frequent terms from the tweets.\\
We first pre-process the filtered tweet text by using tokenization, case normalization, removing stopwords (which we added certain "twitter-specific" stopwords to the english stopwords list) and punctuation removal. We do not use stemming because we found that this in some cases resulted in a loss of information from the tweets, and a decreased accuracy. We also do not use segmentation because we found out that segmentation did not improve our results, and in some cases decreased accuracy. We assume that this is due to the fact that when we search something in Bing that contains concatenated words (e.g. 'presidentobama'), Bing does a better job segmenting the two words 'president' and 'obama' than the word segment toolkit. \\
We then counted the frequency of unigrams, bigrams, and trigrams that occur in the tweets. Based on this counting, we created separate queries consisting of the following. \{Hashtag itself\}, \{hashtag, most frequent 1-gram\}, \{hashtag, 1st most frequent 1-gram, 2nd most frequent 1-gram\},  \{1st most frequent 1-gram, 2nd most frequent 1-gram\}, and if their frequency is above a certain threshold, the top bigram and trigrams.

Example:
\#Saban14   [saban14, \{saban14, iran\}, \{saban14, iran, clinton\},  \{iran, clinton\}, \{saban forum\}, \{hillary clinton israel\}]

\subsection{Web Search and Selection}
Once the queries are generated, the next module performs separate searches on each of the generated queries using the Bing search API.

Merge the results by weighted voting
Queries in top positions have higher weights
URLs in top positions have higher weights

\subsection{Text Summarization}



\section{Evaluation}
 At this time, we are not aware of any available "gold standard" dataset for hashtag meanings.  In order to compensate, we were forced to create our own test dataset, which consisted of a spreadsheet with about 30 hashtags, and known relevant links to each of those hashtags. We understand that this is a small dataset, and not statistically significant, thus is a weak point of our system. We did, in addition, do many heuristic based tests to see if we were getting relevant results based on the links that the system returned. We observed that in addition to noisy results, the system did often return at least one or two of the most relevant documents to a particular hashtag. The text summarization aspect did not perform very well, and we attribute this to the fact that in order to generate an accurate summary, the top few links returned must be very relevant to the hashtag, otherwise it will be a completely inaccurate summary.\\


\section{Conclusions and Future Work}
The main contribution of our work is a novel method of automatically extracting meaning from hashtags so that users can understand topics and trends more easily on Twitter. 
One aspect we wish to have implemented if we had more time was allowing the user to search for a hashtag from a certain tweet, and incorporating the text from that tweet (if efficient amount of text) more heavily than all of the others into the query generation. For example, if someone inquired about the hashtag \#UVA, our system would be able to return links with information about the University of Virginia. However, if the user was curious about the hashtag \#UVA in the context of a tweet that said something along the lines of 'The cavaliers remain undefeated in basketball this season', then the system should be able to extract words such as 'cavaliers' and 'basketball' in addition to 'UVA', and return links that relate to the University of Virginia basketball team. 

%\subsection{References}
\bibliographystyle{acm}
\bibliography{sigproc}

\end{document}
